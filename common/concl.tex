%% Согласно ГОСТ Р 7.0.11-2011:
%% 5.3.3 В заключении диссертации излагают итоги выполненного исследования, рекомендации, перспективы дальнейшей разработки темы.
%% 9.2.3 В заключении автореферата диссертации излагают итоги данного исследования, рекомендации и перспективы дальнейшей разработки темы.
\fixme{Скопированная новизна}
\begin{enumerate}[beginpenalty=10000] % https://tex.stackexchange.com/a/476052/104425
  \item Разработана асимптотическая теория движения заряженных частиц в режиме экстремальных радиационных потерь. Определены общие свойства движения частиц в таком режиме, существенно отличающиеся от таковых в режиме слабой реакции излучения. Продемонстрирован новый метод получения приближённого решения уравнений движения в различных конфигурациях.
  \item Обнаружен и качественно описан эффект развития самоподдерживающегося квантово-электродинамического каскада в поле, приближенном к полю плоской волны. Разработана аналитическая модель, описывающая развитие такого каскада.
  \item Разработана модель для вычисления параметра разрушения при лобовом столкновении сильноточных пучков ультрарелятивистских частиц с учётом реакции излучения. Достоверность модели подтверждена полноразмерным численным трёхмерным моделированием
  \item С помощью полноразмерного трёхмерного численного моделирования продемонстрирована схема эффективной генерации гамма-излучения при взаимодействии сильноточного пучка ультрарелятивистских электронов с протяжённой плазменной мишенью. Разработана аналитическая модель для вычисления эффективности конверсии энергии пучка в энергию гамма-излучения. Найдены параметры пучка для установки FACET-II, оптимальные с точки зрения генерации гамма-излучения.
  \item Разработана и реализована в коде QUILL альтернативная схема для численного решения уравнений Максвелла на регулярной сетке, отличающаяся существенно подавленной численной черенковской неустойчивостью, и подходящей для моделирования пучков ультрарелятивистских частиц.
\end{enumerate}
