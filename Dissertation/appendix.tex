\chapter{Численное решение релятивистских уравнений движения заряженной частицы}
\label{app:numerical}

Релятивистские уравнения движения электрона во внешнем ЭМ поле с учётом реакции излучения записываются в следующем виде
\begin{gather}
    \label{eq:app/base1}
    \dv{\gamma}{t} = -\vb{vE} - F_\mathrm{rr} v^2 , \\
    \label{eq:app/base2}
    \dv{\vb{p}}{t} = - \vb{E - v\times B} - F_\mathrm{rr}\vb{v},
\end{gather}
где импульс электрона $\vb{p}$ нормирован на $m c$, время $t$~---~на $1/\omega$, электрическое и магнитное поля~---~на $m c \omega / e$, $F_\mathrm{rr}$~---~полная мощность излучения, нормированная $mc^2\omega$.
Для численного решения уравнений~\eqref{eq:app/base1}--\eqref{eq:app/base2} без учёта реакции излучения ($F_\mathrm{rr}=0$) существует несколько методов, таких как схема Бориса (Boris)~\cite{boris1970relativistic}, схема Вэя (Vay)~\cite{Vay08} и схема Игэры-Карри (Higuera-Cary, HC)~\cite{higuera2017structure}. 
Последняя из этих схем наиболее точно сохраняет Гамильтониан системы (а значит и фазовый объём), поэтому именно она используется в данной работе при решении уравнений движения одной частицы в заданных полях.
Явный вид данной схемы записывается в следующем виде
\begin{gather}
    \vb{r}_{i+1} = \vb{r}_i + \Delta t \frac{\vb{p}_{i+1/2}}{\gamma_{i+1/2}}, \\
    \vb{p}_{i+1/2} = \tilde{\vb{p}} - \frac{\hat{\vb{p}} }{\hat{\gamma}} \times \symbf{\beta},
\end{gather}
где $\tilde{\vb{p}} = \vb{p}_{i-1/2} - \symbf{\epsilon}$, $\tilde{\gamma} = \sqrt{1 + \tilde{\vb{p}}\cdot\tilde{\vb{p}}}$, $\symbf{\epsilon} = \vb{E}\Delta t / 2$, $\symbf{\beta} = \vb{B} \Delta t / 2$, a $\hat{\vb{p}}$ и $\hat{\gamma}$~---~промежуточные значения импульса и энергии электрона соответственно, вычисляемые следующим образом
\begin{gather}
    \hat{\gamma}^2 = \frac{1}{2}\left( \tilde{\gamma}^2 - \beta^2 + \sqrt{\left( \tilde{\gamma}^2 - \beta^2 \right)^2 + 4 \left( \beta^2 + {\left|\symbf{\beta}\cdot\tilde{\vb{p}}\right|}^2 \right)} \right), \\
    \hat{\vb{p}} = \left( \tilde{\vb{p}} - \frac{\symbf{\beta} \times \tilde{\vb{p}}}{\hat{\gamma}} + \symbf{\beta} \frac{\symbf{\beta}\cdot\tilde{\vb{p}}}{\hat{\gamma}^2} \right) \left( 1 + \frac{\beta^2}{\hat{\gamma}^2} \right)^{-1}.
\end{gather}
Отметим, что координаты (и ЭМ поля) и импульсы электрона определены в моменты времени, смещённые на половину шага $\Delta t$ относительно друг друга, что объясняет полуцелые индексы у импульсов.

Для учёта реакции излучения используется два различных подхода.
В первом, полуклассическом подходе (обозначается на рисунках аббревиатурой LL), радиационное трение считается непрерывной силой, которая добавляется в численную схему по методу Эйлера следующим образом: сначала делается шаг по схеме Игеры-Кары и вычисляется импульс $\vb{p}_{\mathrm{HC}, i+1/2}$, затем конечный импульс электрона определяется следующим образом
\begin{gather}
    \vb{p}_{i+1/2} = \vb{p}_{\mathrm{HC}, i+1/2} - \Delta t F_\mathrm{rr} (\bar{\chi}) \frac{\bar{\vb{p}}}{\bar{\gamma}} , \\
    \bar{\chi} = \frac{1}{a_\mathrm{S}}\sqrt{\left( \bar{\gamma} \vb{E} + \bar{\vb{p}}\times\vb{B} \right)^2 - \left( \bar{\vb{p}} \cdot E \right)^2}, \\
    \bar{\vb{p}} = \frac{\vb{p}_{i-1/2} + \vb{p}_{\mathrm{HC}, i+1/2}}{2}, \\
    \bar{\gamma} = \sqrt{1 + \bar{\vb{p}}\cdot\bar{\vb{p}}},
\end{gather} 
где $a_\mathrm{S} = mc^2 / \hbar\omega$~---~нормированное критическое поле Заутера-Швингера.
Во втором подходе (обозначается на рисунках аббревиатурой MC) учитывается квантовая (стохастическая) природа излучения с помощью метода Монте-Карло схожим образом с методом, реализованным в комплексе QUILL~\cite{QUILL, elkina2011qed}: на каждом шаге разыгрываются два равномерно распределённых в интервале $[0, 1]$ случайных числа $r_0$ и $r_1$; затем вычисляется вероятность $W$ излучения электроном фотона с энергией $r_0 \bar{\gamma}$ за промежуток времени $\Delta t$; если выполняется неравенство $r_1 < W$, то от конечного импульса электрона $\vb{p}_{\mathrm{HC}, i+1/2}$ отнимается величина $r_0 \bar{\vb{p}}$, в противном случае конечный импульс электрона не изменяется.
Величина $W$ вычисляется следующим образом~\cite{Baier98}
\begin{gather}
    W = \Delta t \frac{ \alpha a_\mathrm{S}}{\sqrt{3}\pi\bar{\gamma}} \left( \frac{r_0^2 - 2 r_0 + 2}{1 - r_0}K_{2/3}( y ) - \int\limits_y^{+\infty} K_{1/3}(x)\dd x \right), \\
    y = \frac{r_0}{1 - r_0}\frac{2}{3 \bar{\chi}},
\end{gather}
где $K_\nu(x)$~---~модифицированная функция Бесселя второго рода.

Другие различные дифференциальные уравнения в данной работе решаются с помощью адаптивного метода Рунге-Кутты 8-го порядка (DOP853~\cite{hairer1993solving}), если не указано другое.